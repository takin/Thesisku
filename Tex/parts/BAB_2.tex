\chapter{TINJAUAN PUSTAKA}
\section{Tinjauan Pustaka}
Penelitian yang berfokus pada sistem QA telah banyak dilakukan oleh peneliti-peneliti sebelumnya seperti terlihat dalam Tabel 1, baik dalam bidang open-domain maupun closed-domain. Open-domain QA ditujukan untuk menjawab pertanyaan-pertanyaan umum sedangkan closed-domain QA ditujukan untuk area domain tertentu seperti bidang kesehatan, pemerintahan, musik, prakiraan cuaca dan lain sebagainya. Tujuan dari pengembangan QA berbasis closed-domain adalah untuk mendapatkan sistem yang memiliki knowledge yang lebih spesifik sehingga diharapkan mampu menjawab pertanyaan-pertanyaan secara lebih spesifik pada domain tertentu.

Menurut \citet{zadeh} meskipun algoritma mesin pencari yang ada saat ini seperti Google, Yahoo dan Bing telah mengalami perkembangan yang cukup signifikan namun tetap memiliki keterbatasan dalam hal memaknai sebuah query pencarian, hal ini dikarenakan keterbatasan pada pengetahuan (knowledge) dari sumber pencariannya. Pengetahuan didapatkan dari pengalaman, pembelajaran dan komunikasi yang dilakukan oleh manusia.

Penelitian tentang QA dalam bidang kimia yang dilakukan oleh merepeseresentasi konsep pengetahuan tentang kimia dengan menggunakan metode F-Logic (Frames-Logic), dimana pembentukan kelas dan sub-kelas dimodelkan dalam bentuk isa-hierarchy. Justifikasi kebenaran dari jawaban yang dihasilkan diproses pada saat query dijalankan, inference-engine akan menghasilkan file-log berupa proof-tree untuk setiap jawaban yang diberikan. File proof-tree ini kemudian dijadikan masukan pada proses inferencing tahap kedua yang kemudian menghasilkan jawaban dalam bentuk bahasa alami.

Sistem QA juga dimungkinkan untuk menggabungkan ontologi dengan sumber pengetahuan lainnya untuk dijadikan sebagai basis pengetahuan, hal ini cukup bagus untuk diterapkan terutama dalam bidang open-domain dimana apabila ontologi tidak dapat memberikan jawaban yang revelan, maka sistem dapat mencari sumber lainnya seperti halaman web ataupun data warehouse. Sistem seperti ini telah dilakukan oleh \citet*{guo_zhang}, dimana ia menggunakan tiga buah sumber pengetahuan yaitu ontologi, dokumen warehouse serta halaman web. Peratama-tama sistem akan mencari jawaban dalam ontologi, apabila jawaban tidak ditemukan maka sistem akan melakukan pencarian pada halaman web.

Penelitian lain dalam bidang QA berbasis multi-ontologi juga pernah dilakukan oleh \citet*{lopez} dengan menghasilkan sistem PowerAqua. Sistem ini menggunaan multi-ontology sebagai basis pengetahuannya. Untuk menentukan ontologi yang relevan dengan pertanyaan yang diberikan, mereka  mengembangkan algoritma sendiri yang diberi nama PowerMap. Tahapan pemilihan ontologi yang menjadi sumber pengetahuan serta pemilihan jawaban dilakukan oleh PowerMap dimana algoritma ini akan mentranslasi terminologi yang dimaksudkan oleh pengguna ke dalam berbagai terminologi standar ontologi. Fungsi utama PowerMap adalah untuk mencari dan meng-index ontologi yang sesuai dengan pertanyaan. Proses filtering kata inputdan menggunakan memanfaatkan ontologi yang disediakan oleh WordNet sebagai acuan.

Sistem QA yang memanfaatkan ontologi pengethauan umum seperti YAGO juga telah dikembangkan oleh \citet*{moussa_kader} melalui sistem QASYO, dimana QA ini bersifat open-domain karena memanfaatkan ontologi YAGO yang disediakan oleh DBPedia sebagai basis pengetahuannya. YAGO sendiri adalah merupakan ontologi yang menggabungkan WordNet sebagai basis hirarki konseptualnya serta Wikipedia sebagai sumber pengetahuan faktanya. Secara umum Qasyo menggunakan empat tahapan proses yaitu question classifier, linguistic component, query generator dan query processor. Apabila pertanyaan yang diberikan memiliki jawaban, maka sistem akan memberikan jawaban dalam bentuk bahasa alami, namun jika jawaban tidak ditemukan maka sistem akan memberikan jawaban sederhana ``Don't Know''.

Beberapa peneliti di Indonesia juga telah melakukan penelitian dalam bidang QA yang juga memanfaatkan ontologi sebagai basis pengetahuannya seperti yang dilakukan oleh \citet{bendi} dengan menggunakan ontologi tunggal sebagai basis pengetahuannya serta menggunakan input pertanyaan berupa bahasa alami namun dengan pola kalimat yang sudah ditentukan sebelumnya. Sistem dibangun dengan menggunakan JSP serta Jena sebagai API-nya.

Berbeda dengan Bendi, penelitian yang dilakukan oleh \citet{suryawan} menggunakan dua buah ontologi sebagai sumber pengetahuannya yaitu ontolingua dan ontopustaka. Meski menggunakan dua buah ontologi, namun kedua ontologi ini memiliki fungsi yang berbeda dimana ontolingua digunakan sebagai basis pengetahuan lingistik untuk memproses pola kalimat pertanyaan, sedangkan ontoputaka merupakan ontologi utama yang menyimpan pegetahuan sebenarnya yaitu pengetahuan tentang perpustakaan, sehingga dapat dikatakan penelitian yang dilakukan oleh Suryawan ini merupakan sistem QA dengan ontologi tunggal dengan tipe closed-domain.

Penelitian lain yang dilakukan oleh \citet*{marriot} menggunakan multi-ontologi sebagai basis pengetahuannya. Penelitian ini berfokus pada closed-domain dimana model sistem mengambil contoh pada sebuah perusahaa otomotif yang memiliki berbagai data terpisah seperti data karyawan, produksi dan lain sebagainya. Data ini kemudian dijadikan sebagai sebuah knowledge source. Berbagai sumber pengetahuan tersebut kemudian dibentuk ontologi lokal yang kemudian di gabungkan menjadi sebuah ontologi inti. Ontologi ini kemudian dijadikan sebagai basis pengetahuan utama yang digunakan oleh sistem. Namun berbeda dengan penelitian-penelitian lainnya, dimana Marriot et al tidak menggunakan sistem question answering namun menyajikannya dalam bentuk web portal.

Kecepatan proses pencarian data dari berbagai sumber pengetahuan dapat ditingkatkan salah satunya adalah dengan menerapkan konsep yang diajukan oleh \citet*{vargas_motta} yaitu dengan cara membuat meta-ontologi, dimana ontologi ini memuat informasi atau snapshot dari masing-masing ontologi yang menjadi basis pengetahuan sistem. Dengan metode ini maka proses parsing query akan mejadi lebih efisien karena langsung dilakukan terhadap ontologi yang berkaitan tanpa harus melakukan pencarian pada ontologi lainnya.

Proses merging ontologi dengan cara manual memiliki beberapa kendala seperti lamanya waktu yang dibutuhkan terutama pada ontologi yang berukuran cukup besar dan kompleks. Kendala lainnya adalah kemungkinan inkonsistensi dari ontologi yang dihasilkan. Oleh karena itu beberapa penelitian dalam bidang ontology merging pernah dilakukan oleh beberapa peneliti terdahulu dengan tujuan untuk memudahkan proses merging seperti yang dilakukan oleh \citet*{noy_mussen} dengan mengmbangkan tool SMART dimana tool ini bersifat plugin bagi perangkat lunak Protege. 

Penelitian yang cukup baru dalam bidang merging dilakukan oleh \citet*{stumme_maedche} dimana penelitian ini menawarkan metode baru dalam melakukan proses merging yang disebut dengan FCA-Merge. Metode ini menggunakan pendekatan teknik Bottom-Up dalam membangun ontologi.